\documentclass{beamer}\usepackage[]{graphicx}\usepackage[]{color}
%% maxwidth is the original width if it is less than linewidth
%% otherwise use linewidth (to make sure the graphics do not exceed the margin)
\makeatletter
\def\maxwidth{ %
  \ifdim\Gin@nat@width>\linewidth
    \linewidth
  \else
    \Gin@nat@width
  \fi
}
\makeatother

\definecolor{fgcolor}{rgb}{0.345, 0.345, 0.345}
\newcommand{\hlnum}[1]{\textcolor[rgb]{0.686,0.059,0.569}{#1}}%
\newcommand{\hlstr}[1]{\textcolor[rgb]{0.192,0.494,0.8}{#1}}%
\newcommand{\hlcom}[1]{\textcolor[rgb]{0.678,0.584,0.686}{\textit{#1}}}%
\newcommand{\hlopt}[1]{\textcolor[rgb]{0,0,0}{#1}}%
\newcommand{\hlstd}[1]{\textcolor[rgb]{0.345,0.345,0.345}{#1}}%
\newcommand{\hlkwa}[1]{\textcolor[rgb]{0.161,0.373,0.58}{\textbf{#1}}}%
\newcommand{\hlkwb}[1]{\textcolor[rgb]{0.69,0.353,0.396}{#1}}%
\newcommand{\hlkwc}[1]{\textcolor[rgb]{0.333,0.667,0.333}{#1}}%
\newcommand{\hlkwd}[1]{\textcolor[rgb]{0.737,0.353,0.396}{\textbf{#1}}}%

\usepackage{framed}
\makeatletter
\newenvironment{kframe}{%
 \def\at@end@of@kframe{}%
 \ifinner\ifhmode%
  \def\at@end@of@kframe{\end{minipage}}%
  \begin{minipage}{\columnwidth}%
 \fi\fi%
 \def\FrameCommand##1{\hskip\@totalleftmargin \hskip-\fboxsep
 \colorbox{shadecolor}{##1}\hskip-\fboxsep
     % There is no \\@totalrightmargin, so:
     \hskip-\linewidth \hskip-\@totalleftmargin \hskip\columnwidth}%
 \MakeFramed {\advance\hsize-\width
   \@totalleftmargin\z@ \linewidth\hsize
   \@setminipage}}%
 {\par\unskip\endMakeFramed%
 \at@end@of@kframe}
\makeatother

\definecolor{shadecolor}{rgb}{.97, .97, .97}
\definecolor{messagecolor}{rgb}{0, 0, 0}
\definecolor{warningcolor}{rgb}{1, 0, 1}
\definecolor{errorcolor}{rgb}{1, 0, 0}
\newenvironment{knitrout}{}{} % an empty environment to be redefined in TeX

\usepackage{alltt}
\usetheme{Boadilla}
\useoutertheme{infolinesdave}
\usepackage{graphicx}
\usepackage{epstopdf}

\newcommand\myheading[1]{%
  \par\bigskip
  {\Large\bfseries#1}\par\smallskip}
\IfFileExists{upquote.sty}{\usepackage{upquote}}{}
\begin{document}

\title{Functions and packages. Tutorial Solutions}
\author{David Matten}
\institute[SACEMA]{\includegraphics[height=1.5cm, width=4.5cm]{pictures/sacema.jpg}}
\date{09 June 2014}


\maketitle


\begin{frame}[fragile]{Tutorial Solutions}

\begin{itemize}
\item Describe the hierarchy of packages, functions and libraries.
\end{itemize}
Libraries contain packages.
Packages are collections of R functions, data, and background code in a well-defined format.

\begin{itemize}
\item List the libraries on your computer.
\end{itemize}
A library is a folder on your computer where packages are stored.
Use the command
\begin{knitrout}
\definecolor{shadecolor}{rgb}{0.969, 0.969, 0.969}\color{fgcolor}\begin{kframe}
\begin{alltt}
\hlkwd{.libPaths}\hlstd{()}
\end{alltt}
\end{kframe}
\end{knitrout}
to find your own.
\end{frame}

\begin{frame}[fragile]{contd.}

\begin{itemize}
\item Give 5 packages that are already installed on your machine.
\end{itemize}
You can use the function:
\begin{knitrout}
\definecolor{shadecolor}{rgb}{0.969, 0.969, 0.969}\color{fgcolor}\begin{kframe}
\begin{alltt}
\hlkwd{installed.packages}\hlstd{()}
\end{alltt}
\end{kframe}
\end{knitrout}
for a complete list.

\begin{itemize}
\item Find and install the ”Xtable” package from CRAN.
\end{itemize}

\begin{itemize}
\item Load any package into your R environment, and make use of some piece of it, without errors.
\end{itemize}

\end{frame}

\begin{frame}[fragile]{contd.}



\begin{itemize}
\item Write your own function to take two numbers, multiply them, and print the result within the function. You do not need to return the result from the function.
\end{itemize}
An example function might be:
\begin{knitrout}
\definecolor{shadecolor}{rgb}{0.969, 0.969, 0.969}\color{fgcolor}\begin{kframe}
\begin{alltt}
\hlstd{f} \hlkwb{<-} \hlkwa{function}\hlstd{(}\hlkwc{first}\hlstd{,} \hlkwc{second}\hlstd{)\{}
\hlstd{z} \hlkwb{<-} \hlstd{first} \hlopt{*} \hlstd{second}
\hlkwd{print}\hlstd{(z)}
\hlstd{\}}
\end{alltt}
\end{kframe}
\end{knitrout}

\begin{itemize}
\item Make sure your function can accept arguments by name.
\end{itemize}
Lets prove to ourselves this works.
\begin{knitrout}
\definecolor{shadecolor}{rgb}{0.969, 0.969, 0.969}\color{fgcolor}\begin{kframe}
\begin{alltt}
\hlkwd{f}\hlstd{(}\hlkwc{first}\hlstd{=}\hlnum{4}\hlstd{,} \hlkwc{second}\hlstd{=}\hlnum{5}\hlstd{)}
\end{alltt}
\begin{verbatim}
## [1] 20
\end{verbatim}
\end{kframe}
\end{knitrout}

\end{frame}

\begin{frame}[fragile]{contd.}

\begin{itemize}
\item Call your function using partial matching.
\end{itemize}
\begin{knitrout}
\definecolor{shadecolor}{rgb}{0.969, 0.969, 0.969}\color{fgcolor}\begin{kframe}
\begin{alltt}
\hlkwd{f}\hlstd{(}\hlkwc{fir}\hlstd{=}\hlnum{2}\hlstd{,} \hlkwc{s}\hlstd{=}\hlnum{7}\hlstd{)}
\end{alltt}
\begin{verbatim}
## [1] 14
\end{verbatim}
\begin{alltt}
\hlkwd{f}\hlstd{(}\hlkwc{s}\hlstd{=}\hlnum{7}\hlstd{,} \hlkwc{fir}\hlstd{=}\hlnum{2}\hlstd{)}
\end{alltt}
\begin{verbatim}
## [1] 14
\end{verbatim}
\end{kframe}
\end{knitrout}


\end{frame}



\begin{frame}[fragile]{end}
Thank you.
\end{frame}


\end{document}
